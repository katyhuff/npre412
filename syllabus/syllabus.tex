\documentclass[11pt, a4paper]{article}
\usepackage[inner=1in,outer=1in,top=1in,bottom=1in]{geometry}
\pagestyle{empty}
\usepackage{placeins}
\usepackage{graphicx}
\usepackage{fancyhdr, lastpage, bbding, pmboxdraw}
\usepackage[usenames,dvipsnames]{color}
\definecolor{darkblue}{rgb}{0,0,.6}
\definecolor{darkred}{rgb}{.7,0,0}
\definecolor{darkgreen}{rgb}{0,.6,0}
\definecolor{red}{rgb}{.98,0,0}
\usepackage[colorlinks,pagebackref,pdfusetitle,urlcolor=darkblue,citecolor=darkblue,linkcolor=darkred,bookmarksnumbered,plainpages=false]{hyperref}
\usepackage{amsmath}
\usepackage{amssymb}

\pagestyle{fancyplain}
\fancyhf{}
\lhead{ \fancyplain{}{\CourseTitle} }
%\chead{ \fancyplain{}{} }
\rhead{ \fancyplain{}{\CourseSemester \CourseYear} }
%\rfoot{\fancyplain{}{page \thepage\ of \pageref{LastPage}}}
\fancyfoot[RO, LE] {page \thepage\ of \pageref{LastPage} }
\thispagestyle{plain}
\usepackage{tabularx}


%%%%%%%%%%%%%%%%%%%%%%%%%%%%%%%%%%%%
\usepackage{xspace}

\newcommand{\CourseNumber}{NPRE412}
\newcommand{\CourseTitle}{Nuclear Power Economics and Fuel Management\xspace}%
\newcommand{\CourseInstructor}{Prof. Kathryn Huff\xspace}%
\newcommand{\CourseSemester}{Spring\xspace}%
\newcommand{\CourseYear}{2021\xspace}%
\newcommand{\CourseDays}{MWF\xspace}%
\newcommand{\CourseStart}{10:00am\xspace}%
\newcommand{\CourseEnd}{10:50am\xspace}%
\newcommand{\CourseInstructorEmail}{kdhuff@illinois.edu}
\newcommand{\HuffOfficeHourPlace}{\url{https://illinois.zoom.us/my.katyhuff}\xspace}
\newcommand{\CourseRoom}{\href{https://illinois.zoom.us/j/81869304430}{Zoom: 818 6930 4430, PW: 412}\xspace}%
\newcommand{\CourseBuilding}{\xspace}%
\newcommand{\CourseUniversity}{University of Illinois, Urbana-Champaign\xspace}%
%\newcommand{\TeachingAssistant}{Gwendolyn Chee\xspace}%
%\newcommand{\TAEmail}{Gwendolyn Chee\xspace}%
%\newcommand{\TAOfficeHourDays}{Wednesdays\xspace}%
%\newcommand{\TAOfficeHourStart}{11am\xspace}%
%\newcommand{\TAOfficeHourEnd}{noon\xspace}%
%\newcommand{\TAOfficeHourPlace}{226 Talbot Laboratory\xspace}
%\newcommand{\Course<++>}{<++>}
%\newcommand{\Course<++>}{<++>}
%%%%%%%%%%%%%%%%%%%%%%%%%%%%%%%%%%%%
\title{\CourseNumber: \CourseTitle\\}
\author{\CourseUniversity}
\date{\CourseSemester \CourseYear}
\begin{document}
\maketitle
%\setlength{\unitlength}{1in}
\renewcommand{\arraystretch}{1.5}
\begin{center}
\begin{table}[h]
\begin{tabularx}{\textwidth}{rXrX}
\hline
\textbf{Instructor:} & \CourseInstructor & \textbf{Time:} & \CourseDays \CourseStart -- \CourseEnd \\
\textbf{Email:} &  \href{mailto:\CourseInstructorEmail}{\CourseInstructorEmail} & \textbf{Place:} & \CourseRoom \CourseBuilding\\
\hline
\end{tabularx}
\end{table}
\end{center}

\paragraph{Course Pages:}
\begin{enumerate}
        \item \url{https://compass2g.illinois.edu}
        \item \url{https://github.com/katyhuff/\CourseNumber}
        \item \url{https://mybinder.org/v2/gh/katyhuff/npre412/master}
        \item \CourseRoom
\end{enumerate}

%\paragraph{TA Office Hours:} The teaching assistant for the course, 
%\TeachingAssistant, will hold office hours \TAOfficeHourDays from 
%\TAOfficeHourStart to \TAOfficeHourEnd in \TAOfficeHourPlace.



\paragraph{Professor Office Hours:} Prof. Huff will hold office hours by 
appointment on Zoom. You can make an appointment at 
\url{katyhuff.youcanbook.me}. Appointments must be booked at least 24 hours 
ahead of time.
%If the door to \HuffOfficeHourPlace is open, Prof. Huff may be available for 
%very brief questions. In that case, feel free to drop by.

\paragraph{Main References:}
A few essential references for this course will be assigned as readings. The 
recommended text for this course is \cite{tsoulfanidis_nuclear_2013}.
\bibliographystyle{unsrt}
\renewcommand{\refname}{\normalfont\selectfont\normalsize}
\bibliography{bibliography}

\paragraph{Objectives:} 

This course will equip students to:

\begin{itemize}
\item Quantify impacts of the nuclear power industry
\item Calculate nuclear fuel cycle and capital costs for thermal and fast reactors.
\item Optimize nuclear fuel management for lowest energy costs and highest system performance.
\item Differentiate among features of fossil fuel systems, fission systems, and controlled thermonuclear fusion systems.
\item Quantiatively analyze nuclear fuel cycle technologies for both once-through and closed strategies.
\item Comparatively assess spent fuel storage, reprocessing, and disposal strategies.
\end{itemize}

\paragraph{Prerequisites:} 
\begin{itemize}
        \item Junior standing is encouraged.
\item NPRE 402 or 247
\end{itemize}

\paragraph{Grading Policy:} Grades will be assigned as a weighted sum of the 
following work.

\begin{table}[h]
\begin{tabularx}{\textwidth}{Xrr}
        \textbf{Work} & \textbf{Weight (Undergraduate)} & \textbf{Weight (Graduate)} \\
\hline
\textbf{Quizzes}           & (10\%)   & (0\%)\\
\textbf{Homework}          & (20\%)   & (30\%)\\
\textbf{Project Proposal}  & (30\%)   & (30\%)\\
\textbf{Long Read}         & (20\%)   & (20\%)\\
\textbf{Final Project}     & (30\%)   & (30\%)\\
\hline
\textbf{Total}             & (100\%) & (100\%)\\
\end{tabularx}
\end{table}

\paragraph{Important Dates:}
\begin{center} \begin{minipage}{3.8in}
\begin{flushleft}
%Midterm      \dotfill 10:00-10:50am, March 13, 2021\\
%Project Deadline \dotfill ~Month Day \\
Final Presentations       \dotfill 8:00-11:00 a.m., Friday, May 14, 2021\\
\end{flushleft}
\end{minipage}
\end{center}

\paragraph{Integrity:} This is an institution of higher learning. You will 
        be swiftly ejected from the course if you are caught undermining its 
                integrity. Note the 
                \href{http://www.provost.illinois.edu/academicintegrity/students.html}{Student's 
                Quick Reference Guide to Academic Integrity} and the 
                \href{http://studentcode.illinois.edu/article1_part4_1-401.html}{Academic 
                Integrity Policy and Procedure}.
\paragraph{Attendance:} Regular attendance is mandatory. Request approval
        for absence for extenuating circumstances prior to absence.
\paragraph{Electronics:} Active participation is essential and expected.
        Accordingly, students must turn off all electronic devices (laptop,
        tablets, cellphones, etc.) during class. Exceptions may be granted for
        laptops if engaging in computational exercises or taking notes.
\paragraph{Collaboration:} Collaboratively reviewing course materials and
        studying for exams with fellow students can be enriching.  This is
                recommended.  However, unless otherwise instructed, homework
                assignments are to be completed independently and materials
                submitted as homework should be the result of one's own
                independent work.
\paragraph{Late Work:} Late work has a halflife of 1 hour. That is,
        adjusted for lateness, your grade $G(t)$ is a decaying percentage of
                the raw grade $G_0$. An assignment turned in $t$ hours late
                will receive a grade according to the following relation:
\begin{align*}
        G(t) &= G_0e^{-\lambda t}
        \intertext{where}
        G(t) &= \mbox{grade adjusted for lateness}\\
        G_0 &= \mbox{raw grade}\\
        \lambda &= \frac{ln(2)}{t_{\frac{1}{2}}} = \mbox{decay constant} \\
        t &= \mbox{time elapsed since due [hours]}\\
        t_{1/2} &= 1 = \mbox{half-life [hours]} \\
\end{align*}
\paragraph{COVID} This course will take place online. If any components of 
        the course are made available in person, they will be optional. 
                Following University policy, all students are required to 
                engage in appropriate behavior to protect the health and safety 
                of the community, including wearing a facial covering properly, 
                maintaining social distance (at least 6 feet from others at all 
                times), disinfecting the immediate seating area, and using hand 
                sanitizer. Students are also required to follow the campus 
                COVID-19 testing protocol.

                Students who feel ill must not come to in-person class 
                activities. In addition, students who test positive for 
                COVID-19 or have had an exposure that requires testing and/or 
                quarantine must not attend class.  The University will provide 
                information to the instructor, in a manner that complies with 
                privacy laws, about students in these latter categories. These 
                students are judged to have excused absences for the class 
                period and should contact the instructor via email about making 
                up the work.  

                Students who fail to abide by these rules will first be asked 
                to comply; if they refuse, they will be required to leave the 
                classroom immediately. If a student is asked to leave the 
                classroom, the non- compliant student will be judged to have an 
                unexcused absence and reported to the Office for Student 
                Conflict Resolution for disciplinary action. Accumulation of 
                non-compliance complaints against a student may result in 
                dismissal from the University.  

\paragraph{Emergency response recommendations} can be found at 
        \href{http://police.illinois.edu/emergency-preparedness/}{this website}. 
        I encourage you to review this website and the campus building 
        \href{http://police.illinois.edu/emergency-preparedness/building-emergency-action-plans/}{floor 
        plans website} within the first 10 days of class.  If you want to 
        better prepare yourself for any of these situations, visit 
        \url{police.illinois.edu/safe}. Remember you can sign up for emergency 
        text messages at \url{emergency.illinois.edu}.
\paragraph{Make-up Work:} There will be no negotiation about late work 
        except in the case of absence documented by an absence letter from the 
        Dean of Students. The university policy for requesting such a letter 
        is in 
        \href{http://studentcode.illinois.edu/article1_part5_1-501.html}{the 
        Student Code}. I am sensitive to the global pandemic situation and 
        COVID-related challenges will be accomodated. Please note that such a 
        letter is appropriate for many types of conflicts, but that religious 
        conflicts require special early handling. In accordance with university 
        policy, students seeking an excused absence for religious reasons 
        should complete the Request for Accommodation for Religious Observances 
        Form, which can be found on the Office of the Dean of Students website. 
        The student should submit this form to the instructor and the Office of 
        the Dean of Students by the end of the second week of the course to 
        which it applies.

\paragraph{Grade Disputes:} It is important that you understand and agree
        with the grade you receive on assignments and exams. If you would like
        to dispute your score, you must send an explanation by email to Prof.
        Huff within one week of recieving the grade.
        \textbf{Do not expect me to regrade anything while in conversation with
        you} as that would not be fair to the other students in the class, whose
        homeworks were graded without them present.  If you request a regrade,
        be aware that the entire assignment will be regraded and is subject to
        double-jeopardy: it is possible that your score will go down.
        Regrade requests should be based on an error on my part (e.g., adding
        up the points incorrectly) or what you suspect is a misunderstanding of
        your work (e.g., arriving at the correct answer using an unexpected
        technique). Regrade requests that argue with the rubric (e.g., ``this is
        wrong, but you took too many points off'') will be returned without
        consideration.
        \textbf{Your work should stand alone.} If an assignment is disorganized or
        ambiguous, and requires an extensive explanation to the grader, you
        will likely still lose points. The homeworks not only evaluate your
        understanding of the material - they also evaluate your ability to
        communicate that understanding clearly and concisely.
\paragraph{Anti-Racism and Inclusivity Statement} The Grainger College of 
        Engineering is committed to the creation of an anti-racist, inclusive 
        community that welcomes diversity along a number of dimensions, 
        including, but not limited to, race, ethnicity and national origins, 
        gender and gender identity, sexuality, disability status, class, age, 
        or religious beliefs. The College recognizes that we are learning 
        together in the midst of the Black Lives Matter movement, that Black, 
        Hispanic, and Indigenous voices and contributions have largely either 
        been excluded from, or not recognized in, science and engineering, and 
        that both overt racism and micro-aggressions threaten the well-being of 
        our students and our university community.
        The effectiveness of this course is dependent upon each of us to create 
        a safe and encouraging learning environment that allows for the open 
        exchange of ideas while also ensuring equitable opportunities and 
        respect for all of us. Everyone is expected to help establish and 
        maintain an environment where students, staff, and faculty can 
        contribute without fear of personal ridicule, or intolerant or 
        offensive language. If you witness or experience racism, 
        discrimination, micro-aggressions, or other offensive behavior, you are 
        encouraged to bring this to the attention of the course director if you 
        feel comfortable. You can also report these behaviors to the 
        \href{https://bart.illinois.edu/}{Bias Assessment and Response Team 
        (BART)}.  Based on your report, BART members will follow up and reach 
        out to students to make sure they have the support they need to be 
        healthy and safe. If the reported behavior also violates university 
        policy, staff in the Office for Student Conflict Resolution may respond 
        as well and will take appropriate action.  

\paragraph{Disability-Related Accommodations}
To obtain disability-related academic adjustments and/or auxiliary aids, 
students with disabilities must contact the course instructor and the 
Disability Resources and Educational Services (DRES) as soon as possible. To 
contact DRES, you may visit 1207 S. Oak St., Champaign, call 333-4603, e-mail 
disability@illinois.edu or go to 
\href{https://www.disability.illinois.edu}{disability.illinois.edu}.  If you 
are concerned you have a disability-related condition that is impacting your 
academic progress, there are academic screening appointments available that can 
help diagnosis a previously undiagnosed disability. You may access these by 
visiting the DRES website and selecting “Request an Academic Screening” at the 
bottom of the page.

\paragraph{Family Educational Rights and Privacy Act (FERPA)}
Any student who has suppressed their directory information pursuant to Family 
Educational Rights and Privacy Act (FERPA) should self-identify to the 
instructor to ensure protection of the privacy of their attendance in this 
course. See \href{https://registrar.illinois.edu/academic-records/ferpa/}{this 
link} for more information on FERPA.

\paragraph{Religious Observances}
Illinois law requires the University to reasonably accommodate its students' 
religious beliefs, observances, and practices in regard to admissions, class 
attendance, and the scheduling of examinations and work requirements. You 
should examine this syllabus at the beginning of the semester for potential 
conflicts between course deadlines and any of your religious observances. If a 
conflict exists, you should notify your instructor of the conflict and follow 
the procedure at 
\href{https://odos.illinois.edu/community-of-care/resources/students/religious-observances/}{this 
link} to request appropriate accommodations. This should be done in the first two weeks of classes.

\paragraph{Sexual Misconduct Reporting Obligation:} The University of Illinois is committed to combating sexual misconduct. Faculty and staff members are required to report any instances of sexual misconduct to the University’s Title IX Office. In turn, an individual with the Title IX Office will provide information about rights and options, including accommodations, support services, the campus disciplinary process, and law enforcement options.

A list of the designated University employees who, as counselors, confidential 
advisors, and medical professionals, do not have this reporting responsibility 
and can maintain confidentiality, can be found 
\href{https://wecare.illinois.edu/resources/students/#confidential}{here}.
Other information about resources and reporting is available here: wecare.illinois.edu.



\paragraph{Other Resources:}
University students typically experience a wide range of stressors during their
time on campus. Accordingly, campus resources exist to help students manage
stress levels, mental health, physical health, and emergencies while navigating
this environment. I hope you will take advantage of these campus resources as
soon as they can be of help.

\begin{itemize}
\item \href{https://campusrec.illinois.edu/}{The Campus Recreational Centers}
\item \href{http://counselingcenter.illinois.edu/}{The Counselling Center}
\item \href{https://mckinley.illinois.edu/}{The McKinley Health Clinic}
\item \href{http://www.mckinley.illinois.edu/medical-services/mental-health}{The McKinley Mental Health Clinic}
\item \href{https://odos.illinois.edu/community-of-care/emergency-dean/}{The Emergency Dean}
\end{itemize}


\pagebreak
\FloatBarrier
\renewcommand{\arraystretch}{1}
\begin{table}[h]
\begin{center}
\begin{tabular}{lllcllll}
\multicolumn{8}{c}{\textbf{Course Schedule:}\textit{ Note that this schedule is subject to change}}\\
&&&&&&&\\
\textbf{Date} & \textbf{Week} & \textbf{Day} & \textbf{Unit} & \textbf{Chap.} & \textbf{Quiz} & \textbf{HW} & \textbf{HW}\\
 &  &  &  &  &                                                                                          & \textbf{Given} & \textbf{Due}\\
\hline
\hline
01-25 & 1 & M & Intro      & 1 &           &      &    \\
01-27 & 1 & W & Overview   & 1 &           &      &    \\
01-29 & 1 & F & Economics  & 8 &           &  HW2 & HW1\\
02-01 & 2 & M & Economics  & 8 &        Q1 &      &    \\
02-03 & 2 & W & Economics  & 8 &           &      &    \\
02-05 & 2 & F & Economics  & 8 &           &  HW3 & HW2\\
02-08 & 3 & M & Economics  & 8 &        Q2 &      &    \\
02-10 & 3 & W & Mining \& Milling & 2 &    &      &    \\
02-12 & 3 & F & Mining \& Milling & 2 &    &  HW4 & HW3\\
02-15 & 4 & M & Mining \& Milling & 2 & Q3 &      &    \\
02-17 & 4 & W & \textbf{No Class} &  &    &      &    \\
02-19 & 4 & F & Conversion & 3 &           &  HW5 & HW4\\
02-22 & 5 & M & Enrichment & 3 &        Q4 &      &    \\
02-24 & 5 & W & Enrichment & 3 &           &      &    \\
02-26 & 5 & F & Enrichment & 3 &           &  HW6 & HW5\\
03-01 & 6 & M & Fuel Fabrication & 4 &  Q5 &      &    \\
03-03 & 6 & W & Fuel Fabrication & 4 &     &      &    \\
03-05 & 6 & F & Fuel Fabrication & 4 &     &  HW7 & HW6\\
03-08 & 7 & M & Reactors & 5 &          Q6 &      &    \\
03-10 & 7 & W & Reactors & 5 &             &      &    \\
03-12 & 7 & F & Reactors & 5 &             &  HW8 & HW7\\
03-15 & 8 & M & Reactors & 5 &          Q7 &      &    \\
03-17 & 8 & W & Fuel In-Core & 6 &             &      &     \\
03-19 & 8 & F & Fuel In-Core & 6 &         &  HW9 & HW8 \\
03-22 & 9 & M & Fuel In-Core & 6 &      Q8 &      &   \\
03-24 & 9 & W & \textbf{No Class} &  &    &       &     \\
03-26 & 9 & F & Reprocessing & 7 &         &  HW10 & HW9 \\
03-29 & 10 & M & Reprocessing & 7 &     Q9 &       &     \\
03-31 & 10 & W & Reprocessing & 7 &        &       &     \\
04-02 & 10 & F & Reprocessing & 7 &        &  HW11 & HW10\\
04-05 & 11 & M & Reprocessing & 7 &    Q10 &       &     \\
04-07 & 11 & W & Reprocessing & 7 &         &       &    \\
04-09 & 11 & F & HLW & 9 &                 &  HW12 & HW11\\
04-12 & 12 & M & HLW & 9 &             Q11 &       &     \\
04-14 & 12 & W & HLW & 9 &   &       &     \\
04-16 & 12 & F & HLW & 9 &                 &  HW13 & HW12\\
04-19 & 13 & M & LLW & 10 &             Q12 &       &     \\
04-21 & 13 & W & LLW & 10 &                &       &     \\
04-23 & 13 & F & LLW & 10 &                &  HW14 & HW13\\
04-26 & 14 & M & Nonproliferation & 11 & Q13 &     &     \\
04-28 & 14 & W & Nonproliferation & 11 &   &       &     \\
04-30 & 14 & F & Environment & 12 &        &       & HW14\\
05-03 & 15 & M & Environment & 12 &    Q14 &       &     \\
05-05 & 15 & W & Environment & 12 &        &       &     \\
05-14 & 16 & F & \textbullet~\textbf{Presentations} \textbullet &  &  &  & \\
\end{tabular}
\end{center}
\end{table}
\FloatBarrier
\end{document}


%%%%%% THE END 
